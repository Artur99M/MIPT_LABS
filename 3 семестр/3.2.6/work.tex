\section{Ход работы}
\subsection{Измерения}

1. Настроен осветитель гальванометра: на шкале проявилась четкая вертикальная риска. Делитель установлен на небольшое выходное напряжение ($\frac{R_1}{R_2} = \frac{1}{2000}$), а магазин напряжений на $R = 50$ кОм. Напряжение на источнике питания $U_0 = (1,26 \pm 0,02)$ В.

2. Цепь собрана согласно рис. \ref{pic:p2}. Ключи $K_2$ и $K_3$ разомкнуты. Установка включена в сеть, ключ $K_3$ замкнули, через некоторое время замкнули $K_2$.

3. Измерена зависимость отклонения зайчика $x$ от сопротивления магазина $R$. Результаты представлены в таблице \ref{tab:x(R)}. Погрешность измерений для x будем считать одно деление шкалы, то есть $\Delta x = 0,1$ см.

\begin{table}[h]
    \centering
    \begin{tabular}{|c|c|c|c|}
        \hline
        $x$, см & $R$, кОм & $I$, нА & $\Delta I$, нА\\
        \hline
        23 & 13 & 468 & 7\\
        \hline
        16,6 & 18 & 341 & 5\\
        \hline
        13,1 & 23 & 268 & 4\\
        \hline
        10,7 & 28 & 221 & 4\\
        \hline
        9,2 & 33 & 188 & 3\\
        \hline
        8,1 & 38 & 164 & 3\\
        \hline
        7,3 & 43 & 145 & 2\\
        \hline
        6,2 & 48 & 130 & 2\\
        \hline
        5,6 & 53 & 118 & 21\\
        \hline
    \end{tabular}
    \caption{Зависимость x(R)}
    \label{tab:x(R)}
\end{table}

4. Вернули значение к 13 кОм. Наблюдались свободные колебания. Рассмотрели 3 последовательных колебания и посчитали период колебаний $T_0$.
\[T_0 = \frac{16,1 \pm 0,3}{3} \text{ с}= (5,4 \pm 0,1) \text{ с}\]

5. Подобраны наибольшее значение, при котором зайчик не переходит через положение равновесия при свободных коолебаниях. Этим значением оказалось $R_\text{кр} = 7200$ Ом.

6. Посчитаны значения логарифмического декремента затухания $\Theta$ в зависимости от сопротивления магазина $R$. Данные представлены в таблице \ref{tab:Theta(R)}.
Формула для $\Delta\Theta$: $\Delta\Theta = \frac{\Theta\Delta x}{x^2_2}\sqrt{ x^2_1 + x^2_2}$
\begin{table}[h]
    \centering
    \begin{tabular}{|c|c|c|c|c|}
        \hline
        $R / R_\text{кр}$ & $x_1$, см & $x_2$, см & $\Theta$ & $\Delta\Theta$\\
        \hline
        3 & 9,5 & 1,1 & 2,16 & 1,7\\
        \hline
        3,5 & 9,3 & 1,5 & 1,82 & 0,76\\
        \hline
        4 & 9,1 & 1,7 & 1,67 & 0,53\\
        \hline
        4,5 & 8,9 & 2 & 1,49 & 0,34\\
        \hline
        5 & 8,6 & 2,3 & 1,31 & 0,22\\
        \hline
        6 & 12,9 & 4,1 & 1,14 & 0,09\\
        \hline
        7 & 11,8 & 4 & 1,08 & 0,08\\
        \hline
        8 & 16,8 & 6,5 & 0,95 & 0,04\\
        \hline
        10 & 14,3 & 6,4 & 0,8 & 0,03\\
        \hline
    \end{tabular}
    \caption{Зависимость $\Theta(R)$}
    \label{tab:Theta(R)}
\end{table}

7. Собрана вторая схема. $C = 2$ мкФ, $R_1/R_2 = 1/30$, $a = 133$ см.

8. Получена зависимость $l_\text{max} (R)$. Резульаты представлены в таблице \ref{l(R)}.
\begin{table}[h]
    \centering
    \begin{tabular}{|c|c|}
        \hline
        $l_{max}$, см & $R$, кОм \\
        \hline
        24 & $\infty$ \\
        \hline
        20,6 & 50 \\
        \hline
        19,8 & 40 \\
        \hline
        18,9 & 30 \\
        \hline
        15,6 & 15 \\
        \hline
        13,0 & 10 \\
        \hline
        10,9 & 7,5 \\
        \hline
        9,1 & 5 \\
        \hline
        6,5 & 2,5 \\
        \hline
    \end{tabular}
    \caption{Зависимость $l_{max}(R)$}
    \label{l(R)}
\end{table}

\subsection{Обратботка}

9. Определен ток через гальванометр по формуле (\ref{equ:9}). Данные добалены в таблицу \ref{tab:x(R)}.
\begin{equation}
    I = \frac{R_1}{R_2}\frac{U_0}{R+R_0}.
    \label{equ:9}
\end{equation}
Далее значения нанесены на график на рис. \ref{pic:p4}. Посчитан $C_1$:
\begin{equation*}
    C_1 = (5,40 \pm 0,04) \text{ } \frac{\text{нА}}{\text{мм/м}}.
\end{equation*}

\begin{figure}[h]
    \centering
    \includegraphics*[width = \linewidth]{p4.png}
    \caption{График I(x)}
    \label{pic:p4}
\end{figure}

10. По ниже формуле (\ref{equ:Rkr}) посчитано критическое сопротивление. На рис. \ref{pic:p5} изображен график зависимости $(R+R_0)\left(\sqrt{(\frac{2\pi}{\Theta})^2 + 1}\right)$.
Результат:
\begin{equation*}
    R_\text{кр} = (10 \pm 1) \text{ кОм}.
\end{equation*}
Заметим, что $R_\text{кр}$, которое определено подбором оказалось меньше.

\begin{figure}[h]
    \centering
    \includegraphics*[width = \linewidth]{p5.png}
    \caption{График $(R+R_0)\left(\sqrt{(\frac{2\pi}{\Theta})^2 + 1}\right)$}
    \label{pic:p5}
\end{figure}

11. По формуле (\ref{equ:Ckrq}) расчитан $C_q^\text{кр}$.
 %Для начала расчитаем из графика на рис. \ref{pic:p6} $x^\text{кр}_\text{max}$. Для начала заметим, что зависимость линейная. Значение в точке, где $R = R_\text{кр}$ $x^\text{кр}_\text{max} = \left(13.0\pm 0.7\right)$ см
Так как для $R_\text{кр}$ измерение было проведено, то можно использовать $x_\text{max}^\text{кр} = 13$ см.
\begin{equation*}
    C_q^\text{кр} = (1,71 \pm 0,03) \cdot 10^{-9}\text{ Кл}
\end{equation*}

% \begin{figure}[h]
%     \centering
%     \includegraphics*[width = 0.92\linewidth]{p6.png}
%     \caption{График $\ln (R / R_0) (x_\text{max})$}
%     \label{pic:p6}
% \end{figure}
