\section{Выводы}

 В данной лабораторной работе мы исследовали свободные и вынужденные колебания в электрическом контуре и различными способами находили его добротность. Самый точный способ, конечно же, теоретический. Затем достаточно эффективен способ вычисления через декремент затухания. Фазовая спираль даёт высокую погрешность, поэтому это не очень надежный способ вычисления добротности. Способы вычисления через АЧХ и ФЧХ хороши, если есть специальная программа, позволяющая вычислять ширину резонансной кривой, и хорошо снятые данные. Так как данные для нарастания и затухания были сняты неправильно, на них ориентироваться нельзя. Результаты всех измерений приведены в таблице \ref{tab:7}
 \begin{table}
    \centering
    \begin{tabular}{|c|c|c|c|c|c|}
        \hline
        \multirow{2}{*}{R} & \multicolumn{3}{c}{Свободные колебания} & \multicolumn{2}{|c|}{Вынужденные колебания} \\
        \cline{2-6}
        & $f(L,C,R)$ & $f(\Theta)$ & Спираль & АЧХ & ФЧХ \\
        \hline
        408.25 & $8.5 \pm 0.5$ & $9.9 \pm 0.4$ & $8.2 \pm 0.9$ & $8.1 \pm 0.7$ & $9.1 \pm 0.8$\\
        \hline
        2041.25 & $2.1 \pm 0.7$ & $1.9 \pm 0.1$ & $2.4 \pm 0.2$ & --- & ---\\
        \hline

    \end{tabular}
    \caption{Итоговая таблица}
    \label{tab:7}
 \end{table}
