\section{Ход работы}
\subsection{Измерения}

\textbf{Инструментальные погрешности:}
вольтметр $\Delta U = 0.1$ В, частотометр $\Delta \nu = 1$ Гц, индуктивность $\Delta L = 1$ мГн.

1. На магазине сопротивлений установлено минимальное сопротивление, на магазине индуктивности установлено значение $L = 100$ мГн. Вычисленно значение емкости $C_0 = 1,2$ нФ -- емкость цепи без включенного конденстатора.

2. С помощью изменения емкости измерены периоды. Результаты в табл. \ref{tab:1}.

\begin{table}[h]
    \centering
    \begin{tabular}{|c|c|}
        \hline
        $T$, мкс & $C_\text{магазин}$, мкФ \\
        \hline
        94,2 & 0,001 \\
        \hline
        114	& 0,002 \\
        \hline
        128	& 0,003 \\
        \hline
        145	& 0,004 \\
        \hline
        156	& 0,005 \\
        \hline
        168	& 0,006 \\
        \hline
        190	& 0,008 \\
        \hline
        200	& 0,009 \\
        \hline
    \end{tabular}
    \caption{Зависимость $T(C_\text{магазин})$}
    \label{tab:1}
\end{table}

3. Приняв $L = 100$ мГн, рассчитана $C^*$, при которой собственная частота колебаний цепи $\nu_0 = \frac{1}{2\pi\sqrt{LC^*}}$ равно 6,5 кГц. $C^* = 6$ нФ. Также рассчитано критическое сопротивление $R_{cr} = 2\sqrt{L/C^*} = 8165$ Ом.

4. На магазине установлена емкость 6 нФ. Рассмотрены 2 соседние амплитуды в зависимости от сопротивления $R$ на магазине. Результаты приведены в таблице \ref{tab:2}

\begin{table}[h]
    \centering
    \begin{tabular}{|c|c|c|}
        \hline
        $R$, Ом & $U_m$, В & $U_{m+1}$, В \\
        \hline
        408,25  & 3,16 & 2,18 \\
        \hline
        734,85  & 2,66 & 1,4 \\
        \hline
        1061,45 & 2,16 & 0,88 \\
        \hline
        1388,05 & 1,86 & 0,6 \\
        \hline
        1714,65 & 1,52 & 0,42 \\
        \hline
        2041,25 & 1,26 & 0,28 \\
        \hline
    \end{tabular}
    \caption{$U_m$ и $U_{m+1}$ от R}
    \label{tab:2}
\end{table}

5. На магазине введено значение $408,25$ Ом. На канал 2(Y) подано падение напряжения на резисторе. Результаты измерений приведены в таблице \ref{tab:3}.

\begin{table}[h]
    \centering
    \begin{tabular}{|c|c|}
        \hline
        $R$, Ом & $\Delta$, цена деления \\
        \hline
        408,25 & 0,6 \\
        \hline
        734,85 & 0,4 \\
        \hline
        1061,45 & 0,25 \\
        \hline
        1388,05 & 0,13 \\
        \hline
        1714,65 & 0,10 \\
        \hline
        2041,25 & 0,08 \\
        \hline
    \end{tabular}
    \caption{Свободные колебния на фазовой плоскости}
    \label{tab:3}
\end{table}

6. Генератор переключен в режим подачи синусоидального сигнала. Значение емкости выставлено в $C^*$, а сопротивление $R_1 = 0,05R_{cr}$. Схема собрана как на рис. \ref{pic:p2}. Также найдено значение значение резонансной частоты $\nu_\text{рез} = 5,92$ кГц, в близи которой и будет изменяться частота. Результаты зависимости приведены в таблице \ref{tab:4}.
\begin{figure}[h]
    \centering
    \includegraphics*[width = \linewidth]{p2.png}
    \caption{Схема установки для исследования АЧХ и ФЧХ}
    \label{pic:p2}
\end{figure}
\begin{table}[h]
    \centering
    \begin{tabular}{|c|c|c|}
        \hline
        $\nu$, кГц & $2U_C$, В & $\Delta t$, мкс \\
        \hline
        5,92 & 18,6 & 38,8 \\
        \hline
        5,87 & 18 & 44,4 \\
        \hline
        5,82 & 17,3 & 46,4 \\
        \hline
        5,77 & 16,6 & 51,6 \\
        \hline
        5,72 & 15,8 & 56,4 \\
        \hline
        5,62 & 13,6 & 61,2 \\
        \hline
        5,52 & 11,4 & 66,3 \\
        \hline
        5,42 & 9,7 & 72 \\
        \hline
        5,32 & 8,7 & 75,2 \\
        \hline
        5,22 & 7,5 & 78,8 \\
        \hline
        5,97 & 18,2 & 34,4 \\
        \hline
        6,07 & 17,5 & 27,6 \\
        \hline
        6,17 & 16,2 & 22 \\
        \hline
        6,27 & 14,5 & 17,2 \\
        \hline
        6,37 & 13,3 & 14,6 \\
        \hline
        6,47 & 11,8 & 12,4 \\
        \hline
        6,57 & 10,3 & 10,8 \\
        \hline
        6,67 & 9,52 & 9,2 \\
        \hline
        6,77 & 8,8 & 8,2 \\
        \hline
        6,87 & 8,2 & 6,8 \\
        \hline
    \end{tabular}
    \begin{tabular}{|c|c|c|}
        \hline
        $\nu$, кГц & $2U_C$, В & $\Delta t$, мкс \\
        \hline
        6,02 & 4,58 & 28,8 \\
        \hline
        5,97 & 4,58 & 29,2 \\
        \hline
        5,92 & 4,58 & 31 \\
        \hline
        5,87 & 4,52 & 32,2 \\
        \hline
        5,77 & 4,48 & 33,8 \\
        \hline
        5,67 & 4,38 & 34,6 \\
        \hline
        5,57 & 4,26 & 36,4 \\
        \hline
        5,47 & 4,16 & 39,8 \\
        \hline
        5,37 & 4,02 & 41,4 \\
        \hline
        5,27 & 3,86 & 43,4 \\
        \hline
        6,12 & 4,70 & 25,2 \\
        \hline
        6,22 & 4,70 & 24,2 \\
        \hline
        6,32 & 4,68 & 22,4 \\
        \hline
        6,42 & 4,68 & 19,8 \\
        \hline
        6,52 & 4,68 & 19 \\
        \hline
        6,62 & 4,66 & 18 \\
        \hline
        6,72 & 4,6 & 17,4 \\
        \hline
        6,82 & 4,58 & 15,4 \\
        \hline
    \end{tabular}
    \caption{Исследование резонансных кривых для $R_1 = 0,05R_{cr}$ и $R_2 = 0,25R_{cr}$}
    \label{tab:4}
\end{table}

7. На генераторе установлена резонансная частота. Установлены период повторения сигнала 20 мс, а количеством периодов 15.
Измерены амплитуды колебаний, результаты предсьтавлены в таблице \ref{tab:5}.
\begin{table}[h]
    \centering
    \begin{tabular}{|c|c|c|}
        \hline
        $U_1$, В & $U_2$, В & $\Theta$ \\
        \hline
        20 & 17,4 & 0.14\\
        \hline
        15 & 12,3 & 0.20\\
        \hline
        17 & 14,7 & 0.15\\
        \hline
    \end{tabular}
    \caption{Процессы установления и затухания}
    \label{tab:5}
\end{table}

\subsection{Обработка данных}

8. Пострен график $T_\text{экс}(T_\text{теор})$ из пункта 2, где $T_\text{теор}  = 2\pi\sqrt{LC}$. Его можно увидеть на рис. \ref{pic:p3}. Нетрудно заметить, что теоритические данные практически сошлись с эксперементом. Погрешность получилась около 2\%.

\begin{figure}[h]
    \centering
    \includegraphics*[width = 0.5\linewidth]{p3.png}
    \caption{График $T_\text{экс}(T_\text{теор})$}
    \label{pic:p3}
\end{figure}

9. Построен график $1/\Theta^2 = f(1/R^2)$ для пункта 4. Данный представлена на рис. \ref{pic:p4}. Koэффициент наклона $K = (1.18 \pm 0.07)$ кОм$^2$. Тогда посчитаем $R_{cr} = 6.8 \pm 0.2$ кОм, что не совпадает с теоретическим значением.

\begin{figure}[h]
    \centering
    \includegraphics*[width = 0.7\linewidth]{p4.png}
    \caption{График $1/\Theta^2 = f(1/R^2)$}
    \label{pic:p4}
\end{figure}

10. Вычислена доброность контура для пункта 4:
\[Q(\Theta_{min}) = 8.5 \pm 0.5, \text{      } Q(\Theta_{max}) = 2.1 \pm 0.7.\]

11. Рассчет добротности по спирали на фазовой плоскости. В помощью осциллографа получаем портрет колебаний на фазовой плоскости (в режиме XY), определяем декремент затухания по соседним пересечениям оси X. Так как измерения проводились на глаз, то оценим погрешность $\Delta_U = 0,1$ дел. Результаты представлены в таблице \ref{tab:6}

	\begin{table}[h]
		\centering
		\begin{tabular}{|c|c|c|c|c|c|}
			\hline
			R, Ом & $U_k$, дел & $U_{k + 1}$, дел & $\theta$ & $Q$ & $\Delta_{Q}$  \\ \hline
			408 & 4.4 & 3 & 0.38 & 8.2 & 0.9 \\ \hline
			1714 & 3.8 & 1 & 1.34 & 2.4 & 0.2 \\ \hline
		\end{tabular}
		\caption{Определение добротности по фазовой плоскости}
        \label{tab:6}
	\end{table}

12. Рассчитаны теоретические значения добротности:
\[Q = \frac{\pi}{\Theta} = \frac 12 \sqrt{\frac{4L}{CR^2} - 1},\]
\[Q_1 = 9.9 \pm 0.4,\]
\[Q_2 = 1.9 \pm 0.1.\]

13. Построен график зависимости $U / U_0 = f (\nu / \nu_0)$, где $U_0$ -- напряжение при резонансной частоте $\nu_0$. График представлен на рис. \ref{pic:p5}

\begin{figure}[h]
    \centering
    \includegraphics*[width = 0.6\linewidth]{p5.png}
    \caption{График $U / U_0 = f (\nu / \nu_0)$}
    \label{pic:p5}
\end{figure}

14. По АЧХ определена добротность контура:
\[Q = \frac{\omega_0}{2\Delta\Omega} = 8.1 \pm 0.7.\]

15. Построена ФЧХ (рис. \ref{pic:p6}). По нему определена добротность контура:
\[Q = 9.1 \pm 0.8\]
\begin{figure}[h]
    \centering
    \includegraphics*[width = 0.6\linewidth]{p6.png}
    \caption{ФЧХ}
    \label{pic:p6}
\end{figure}

