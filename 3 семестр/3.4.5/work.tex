\section{Ход работы}

1. Данные, снятые с установки:
\[R_U = 20 \text{ кОм}, \text{    } C_U = 20 \text{ мкФ}, \text{    } R_0 = 0.2 \text{ Ом}.\]

2. Данные с торроидальных образцов:\\
Феррит: \\
$N_0 = 40, \text{     } N_U = 400, \text{     } S = 3 \text{ см}^2, \text{     } 2\pi R  = 25 \text{ см};$\\
Пермалой: \\
$N_0 = 35, \text{     } N_U = 220, \text{     } S = 3.8 \text{ см}^2, \text{     } 2\pi R  = 24 \text{ см};$\\
Кремнистое железо: \\
$N_0 = 40, \text{     } N_U = 400, \text{     } S = 1.2 \text{ см}^2, \text{     } 2\pi R  = 10 \text{ см}.$\\

3. Для каждого образца сфотографируем предельную петлю: Феррит  -- рис. \ref{pic:p1},
Пермаллой -- рис. \ref{pic:p2:1}, Кремнистое железо -- рис. \ref{pic:p2:2}.
\begin{figure}[h]
    \centering
    \includegraphics*[width = 0.5\linewidth]{p1.jpg}
    \caption{Петля гестерезиса феррита}
    \label{pic:p1}
\end{figure}
\begin{figure}[h]
    \centering
    \includegraphics*[width = 0.5\linewidth]{p2.png}
    \caption{Петля гестерезиса пермалоя}
    \label{pic:p2:1}
\end{figure}
\begin{figure}[h]
    \centering
    \includegraphics*[width = 0.5\linewidth]{p3.png}
    \caption{Петля гестерезиса кремнистого железа}
    \label{pic:p2:2}
\end{figure}

4. Для разных токов измерены данные для петель гестерезиса. Все данные предсталены в таблицах \ref{tab:1}, \ref{tab:2}, \ref{tab:3}.
\begin{table}
    \centering
    \begin{tabular}{|c|c|c|c|c|c|c|c|c|}
        \hline
        Ток, мА & Xа & Yа & ЦД x, мВ & ЦД y, мВ & Xкоэр & Yкоэр & ЦД x, мВ & ЦД y, мВ \\
		\hline
		240 & 24 & 23 & 50 & 20 & 4 & 9 & 50 & 20 \\
		\hline
		210 & 22 & 22 & 50 & 20 & 36 & 35 & 5 & 5 \\
		\hline
		180 & 46 & 21 & 20 & 20 & 36 & 34 & 5 & 5 \\
		\hline
		150 & 40 & 20 & 20 & 20 & 31 & 31 & 5 & 5 \\
		\hline
		120 & 33 & 38 & 20 & 10 & 29 & 31 & 5 & 5 \\
		\hline
		90 & 24 & 33 & 20 & 10 & 26 & 28 & 5 & 5 \\
		\hline
		60 & 36 & 25 & 10 & 10 & 25 & 22 & 5 & 5 \\
		\hline
		30 & 34 & 19 & 5 & 5 & 11 & 6 & 5 & 5 \\
		\hline
    \end{tabular}
    \caption{Данные для Феррита}
    \label{tab:1}
\end{table}

\begin{table}
    \centering
    \begin{tabular}{|c|c|c|c|c|c|c|c|c|}
        \hline
        Ток, мА & Xа & Yа & ЦД x, мВ & ЦД y, мВ & Xкоэр & Yкоэр & ЦД x, мВ & ЦД y, мВ \\
		\hline
		170 & 39 & 38 & 20 & 50 & 27 & 36 & 20 & 50 \\
		\hline
		150 & 31 & 35 & 20 & 50 & 25 & 34 & 20 & 50 \\
		\hline
		130 & 24 & 30 & 20 & 50 & 22 & 28 & 20 & 50 \\
		\hline
		110 & 41 & 21 & 10 & 50 & 39 & 20 & 10 & 50 \\
		\hline
		90 & 36 & 28 & 10 & 20 & 32 & 25 & 10 & 20 \\
		\hline
		70 & 30 & 21 & 10 & 10 & 23 & 16 & 10 & 10 \\
		\hline
		50 & 46 & 14 & 5 & 5 & 25 & 7 & 5 & 5 \\
		\hline
    \end{tabular}
    \caption{Данные для Пермалоя}
    \label{tab:2}
\end{table}

\begin{table}
    \centering
    \begin{tabular}{|c|c|c|c|c|c|c|c|c|}
        \hline
        Ток, мА & Xа & Yа & ЦД x, мВ & ЦД y, мВ & Xкоэр & Yкоэр & ЦД x, мВ & ЦД y, мВ \\
		\hline
		300 & 30 & 30 & 50 & 50 & 9 & 14 & 50 & 50 \\
		\hline
		260 & 28 & 28 & 50 & 50 & 42 & 34 & 10 & 20 \\
		\hline
		220 & 23 & 25 & 50 & 50 & 39 & 32 & 10 & 20 \\
		\hline
		180 & 47 & 22 & 20 & 50 & 35 & 30 & 10 & 20 \\
		\hline
		140 & 35 & 18 & 20 & 50 & 30 & 12 & 10 & 20 \\
		\hline
		100 & 48 & 30 & 10 & 20 & 24 & 18 & 10 & 20 \\
		\hline
		60 & 30 & 30 & 10 & 10 & 15 & 16 & 10 & 10 \\
		\hline
    \end{tabular}
    \caption{Данные для Кремнистого железа}
    \label{tab:3}
\end{table}

5. Проведена калибровка горизонтальной оси Y ЭО. Измерения проводились для ЦД 50 и 20 мВ. Измерена двойная амплитуда синусоиды $2y_1 = 3$ см, $2y_2 = 3.2$ см. Эффективные значения напряжения на вольтметре $U_1 = 100$ мВ и $U_2 = 44$ мВ. Погрешностьи примем равными ценам деления: $\sigma_y = 0.2$ см, $\sigma_U = 1$ мВ.
Полученные чувствительности:
\[K_{1y} = (94 \pm 6), \]
\[K_{2y} = (39 \pm 3). \]

6. На вход ячейки подано напряжение 6,3 В. Напряжение на выходе получилось равным 4,24 В по осциллографу и 4,3 по вольтметру. Так точность осциллографа выше, возьмем его. Частота, подаваемая на вход, равна 50 Гц. Погрешность будет только у измерений осциллографа. Она равна 8\%. Итого, получен результат
\[\tau = U_\text{вх}/(\omega U_\text{вых}) = (0.45 \pm 0.04) \text{ мс}.\]
Заметим, что получилось довольно близко к теоретическому результату $\tau = RC = 0.4$ мс.

7. Значения в предельных точках:\\
Феррит\\
\[H = 19 \pm 2\text{ А/м,     } B = 15 \pm 1 \text{ мТл};  \]
Пермаллой\\
\[H = 11 \pm 1\text{ А/м,     } B = 90 \pm 3 \text{ мТл};  \]
Кремнистое железо\\
\[H = 60 \pm 4\text{ А/м,     } B = 125 \pm 5 \text{ мТл}.  \]

8. Вычислены $H_c$ и $B_s$. Данные внесены в таблицу \ref{tab:4}.

\begin{table}
	\centering
	\begin{tabular}{|c|c|c|c|}
	\hline
	Ампл. & Fe-Ni & Fe-Si & Феррит \\
	\hline
	$H_c$, $\frac{A}{\text{м}}$ & $3.9 \pm 0.2$ & $8.0 \pm 0.2$ & $8.2 \pm 0.2$\\
	$B_s$, Тл & $1.04 \pm 0.7$& $0.24 \pm 0.04$& $1.9 \pm 0.2$\\
	% $\mu_\text{нач}$ & & & \\
	% $\mu_{max}$ & & & \\
	\hline
	\end{tabular}
	\caption{}
	\label{tab:4}
\end{table}
