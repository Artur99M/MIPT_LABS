\section{Ход работы}
\subsection{Измерения}

1. Некоторые данные с установки:
\begin{align*}
    m_\text{медь} = 567.5 \pm 0.5 \text{ г}, \\
    m_\text{железо} = 814.8 \pm 0.5 \text{ г}.
\end{align*}

2. График для изменения температуры в каллориметре представлен в приложении на рис. 1.%\ref{fig:Vt}
(По нему можно сопоставить нужные моменты времени)

\subsection{Обработка}

3. Для случая охлаждения построим график в осях $\ln\frac{T_{cool} - T_K}{T - T_K}$, $t$ для пустого каллориметра. В приложении на рис. 2. %\ref{fig:d_empty}
В таком случае получилось $\frac \lambda C = (30 \pm 5) \cdot 10^{-5} \text{ с}^{-1}$.

4. Из уравнения (\ref{T_heat_t}) ясно, что $\lambda$ можно найти по углу наклона прямой $T_{heat}\left(P(1 - e^{- \frac{\lambda}{C} t})\right)$ (рис. 3). Из этой формулы $\lambda = \frac{1}{k}$.

5. Получим окончательные выражения:
\begin{align*}
    \lambda = 0.22 \pm 0.04 \text{ }\frac{\text{Дж}}{\text{К}\cdot\text{с}}, \\
    C_\text{кал} = 0.75 \pm 0.18 \text{ }\frac{\text{кДж}}{\text{К}}.
\end{align*}

6. Аналогично для железа:
\begin{align*}
    \frac \lambda C = (20 \pm 4) \cdot 10^{-5} \text{ с}^{-1}, \\
    \lambda = (0.23 \pm 0.05) \text{ }\frac{\text{Дж}}{\text{К}\cdot\text{с}}, \\
    C = 1 \pm 0.3  \text{ }\frac{\text{кДж}}{\text{К}}, \\
    c_\text{железо} = \frac{C - C_\text{кал}}{m_\text{железо}} = 0.6 \pm 0.3 \text{ } \frac{\text{кДж}}{\text{кг}}. \\
\end{align*}

7. Аналогично для меди:
\begin{align*}
    \frac \lambda C = (20 \pm 3) \cdot 10^{-5} \text{ с}^{-1}, \\
    \lambda = (0.20 \pm 0.04) \text{ }\frac{\text{Дж}}{\text{К}\cdot\text{с}}, \\
    C = 1.0 \pm 0.2  \text{ }\frac{\text{кДж}}{\text{К}}, \\
    c_\text{медь} = \frac{C - C_\text{кал}}{m_\text{медь}} = 0.4 \pm 0.2 \text{ } \frac{\text{кДж}}{\text{кг}}. \\
\end{align*}

8. Теплоемкость калориметра дифферинциальным методом по формуле (16):
\begin{align*}
    C_\text{каллориметр} = 0.627 \pm 0.2 \text{ } \frac{\text{кДж}}{\text{К}} \\
\end{align*}

9. Аналогично для железа:
\begin{align*}
    C = \frac{27.2 \cdot 0.226}{0.01} = 0.614 \pm 0.2 \text{ } \frac{\text{кДж}}{\text{К}} \\
    c_\text{железо} = \frac{C - C_\text{каллориметр}}{m_\text{железо}} < 0,
\end{align*}

10. Аналогично для меди:
\begin{align*}
    C = \frac{27.2 \cdot 0.226}{0.0098} = 0.6 \pm 0.2 \text{ } \frac{\text{кДж}}{\text{К}} \\
    c_\text{медь} = \frac{C - C_\text{каллориметр}}{m_\text{медь}} = 0.0 \pm 0.3 ~\frac{\text{кДж}}{\text{кг}\cdot\text{К}}
\end{align*}
поэтому в данной ситуации посчитано неправильно.
