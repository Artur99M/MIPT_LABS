\section{Ход работы}
\subsection{Измерения}
1. Измерены димаетры труб:

\[d_1 = 3.9 \pm 0.05 \text{ мм},\]
\[d_2 = 3 \pm 0.1 \text{ мм},\]
\[d_3 = 5.25 \pm 0.1 \text{ мм}.\]

2. Измерены давление, температура и влажность окружающего воздуха:
\[P = \text{хз},\]
\[t = 24\text{ }^\circ \text{C},\]
\[\varphi = \text{хз}.\]

3. Расчитана $Q_\text{крит} \approx 4 \text{ л/мин}$.

4. Рассмотрена зависимость изменения давления от расхода вохдуха, то есть с постоянной длиной между точками $l = 50$ см и $d = d_1$
\begin{center}
\begin{tabular}[h]{|c|c|}
   \hline
    $\Delta P$, Па & $Q_\text{ср}$, л/мин\\
   \hline
   68,294277 & 3,11 \\
   \hline
   84,176667 & 3,764 \\
   \hline
   95,29434 & 4,174 \\
   \hline
   111,17673 & 4,78 \\
   \hline
   127,05912 & 5,242 \\
   \hline
   144,529749 & 5,445 \\
   \hline
   162,000378 & 5,728 \\
   \hline
   176,294529 & 5,924 \\
   \hline
   222,35346 & 6,582 \\
   \hline
   49,235409 & 2,275 \\
   \hline
   55,588365 & 2,464 \\
   \hline
   31,76478 & 1,436 \\
   \hline
   103,235535 & 4,48 \\
   \hline
   76,235472 & 3,345 \\
   \hline
   61,941321 & 2,768 \\
   \hline
\end{tabular}
\end{center}

5. Аналогичная зависимость рассмотрена для $d = 5,25$ мм и $l = 50$ см.
\begin{center}
\begin{tabular}[h]{|c|c|}
   \hline
    $\Delta P$, Па & $Q_\text{ср}$, л/мин\\
   \hline
    39,705975 & 5,728 \\
    \hline
    47,64717 & 6,94 \\
    \hline
    55,588365 & 7,486 \\
    \hline
    71,470755 & 7,991 \\
    \hline
    79,41195 & 8,334 \\
    \hline
    87,353145 & 8,55 \\
    \hline
    95,29434 & 8,787 \\
    \hline
    111,17673 & 9,41 \\
    \hline
    119,117925 & 9,744 \\
    \hline
\end{tabular}
\end{center}

6. Теперь изменялись давление и длина при расходе $Q_\text{ср} = 3,336$ л/мин на трубе 3,9 мм.
\begin{center}
\begin{tabular}[h]{|c|c|}
    \hline
    $\Delta P$, Па & $l$, см \\
    \hline
    128,647359 & 90 \\
   \hline
   76,235472 & 50 \\
   \hline
   182,647485 & 120 \\
   \hline
   106,412013 & 70 \\
   \hline
   46,058931 & 30 \\
   \hline
\end{tabular}
\end{center}

7. Аналогичные измерения проведены для трубы 5,25 мм и $Q = 4,5$ л/мин.
\begin{center}
\begin{tabular}[h]{|c|c|}
    \hline
    $\Delta P$, Па & $l$, см \\
    \hline
    57,176604 & 90 \\
    \hline
    77,823711 & 120 \\
    \hline
    47,64717 & 70 \\
    \hline
    27,000063 & 40 \\
    \hline
    20,647107 & 30 \\
    \hline
\end{tabular}
\end{center}

8. Проведены измерения для выявления зависимости $Q(d)$. При этом $\Delta P \approx 31,77$ Па, $l = 40$ см.
\begin{center}
\begin{tabular}[h]{|c|c|}
    \hline
    $d$, мм & $Q$, л/мин \\
    \hline
    5,25 & 5,31 \\
    \hline
    3,9 & 1,751 \\
    \hline
    3 & 1,536 \\
    \hline
\end{tabular}
\end{center}

\subsection{Обработка}

9. Построены графики зависимости для п. 4 и 5:
\begin{figure*}[h]
    \centering
    \includegraphics*[width = 0.8\linewidth]{p1.png}
\end{figure*}

\begin{figure*}[h]
    \centering
    \includegraphics*[width = 0.8\linewidth]{p2.png}
\end{figure*}

Получились уравнения для каждого из случаев (ламинарное течение):
\[Q_1 = 0,0431\Delta P + 0,0101,\]
\[Q_2 = 0,147\Delta P - 0,1261.\]

Получилось
\[\frac{\partial Q_1}{\partial (\Delta P)} = (431 \pm 7) \cdot 10^{-7} \text{ } \frac{\text{м}^3}{\text{Па}\cdot\text{мин}},\]
\[\frac{\partial Q_2}{\partial (\Delta P)} = (144 \pm 3) \cdot 10^{-6} \text{ } \frac{\text{м}^3}{\text{Па}\cdot\text{мин}}.\]

10. По формуле (\ref*{equ:puaz}) посчитаем значения коэффициента динамической вязкости:
\[\eta_1 = (1,6 \pm 0,3) \cdot 10^{-5} \text{ Па}\cdot c,\]
\[\eta_2 = (1,6 \pm 0,5) \cdot 10^{-5} \text{ Па}\cdot c.\]

11. Построены графики зависмости для п. 6 и 7.
\begin{figure*}[h]
    \centering
    \includegraphics*[width = 0.8\linewidth]{p3.png}
\end{figure*}
\begin{figure*}[h]
    \centering
    \includegraphics*[width = 0.8\linewidth]{p4.png}
\end{figure*}

Получились уравнения
\[l_1 = 0,6692\Delta P - 0,2698,\]
\[l_2 = 1,5852\Delta P - 3,0118.\]

Получилось
\[\frac{\partial l_1}{\partial (\Delta P)} = (67 \pm 1) \cdot 10^{-4} \text{ } \frac{\text{м}}{\text{Па}},\]
\[\frac{\partial l_2}{\partial (\Delta P)} = (158 \pm 2) \cdot 10^{-5} \text{ } \frac{\text{м}}{\text{Па}}.\]

12. По результатам п. 8 измерен коэффициент наклона $\ln Q(\ln R)$. Измерения на $d_1$ не учитывалось, так как там было турбулентное течение.
\[\frac{\partial (\ln Q)}{\partial (\ln R)} = 0,5 \text{ }\frac{\text{л}\cdot\text{мм}}{\text{мин}}\]
