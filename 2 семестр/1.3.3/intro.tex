\section{Введение}

\textbf{Цель работы:}
экспериментально исследовать свойства течения газов по тонким трубкам при различных числах Рейнольдса; выявить область применимости закона
Пуазейля и с его помощью определить коэффициент вязкости воздуха.

\textbf{В работе используются:}
система подачи воздуха (компрессор, поводящие трубки); газовый счетчик барабанного типа; спиртовой микроманометр с регулируемым
наклоном; набор трубок различного диаметра с выходами для подсоединения
микроманометра; секундомер.
\subsection{Теоритическая справка}
Работа посвящена изучению течения воздуха по прямой трубе круглого
сечения. Движение жидкости или газа вызывается перепадом внешнего
давления на концах $\Delta P$ трубы, чему в свою очередь препятствуют силы вязкого
(«внутреннего») трения, действующие между соседними слоями жидкости, а
также со стороны стенок трубы.

Сила вязкого трения как в жидкостях, так и в газах описывается законом
\textit{Ньютона}: касательное напряжение между слоями пропорционально перепаду
скорости течения в направлении, поперечном к потоку. В частности, если
жидкость течёт вдоль оси \textit{x}, а скорость течения $v_x(y)$ зависит от координаты \textit{y}, в
каждом слое возникает направленное по \textit{x} касательное напряжение
\begin{equation}
    \tau_{xy} = -\eta \frac{\partial v_x}{\partial y}.
\end{equation}

Характер течения определяется безразмерным параметром задачи -— числом Рейнольдса:
\begin{equation}
    \text{Re} = \frac{\rho ua}{\eta},
\end{equation}
где $u$ -- характерная скорость течения, $a$ -- характерный размер системы.

В целях упрощения теоретической модели течение газа в условиях
эксперимента можно считать несжимаемым, то есть принять плотность среды
постоянной: $\rho$ = const. Для газов такое приближение допустимо, если
относительный перепад давления в трубе мал $\Delta P\ll P $, а скорость течения
значительно меньше скорости звука (число Маха много меньше единицы). В нашем
опыте максимальная разность давлений составляет $\sim$ 30 см водного столба
(3 кПа), что составляет $\sim$ 3\% от атмосферного давления, причем в «рабочем»
(ламинарном) режиме перепад в несколько раз меньше ($\sim$ 5 ÷ 10 см вод. ст.).

Формула Пуазейля для определения расхода жидкости или газа при ламинарном течении:
\begin{equation}
    Q = \frac{\pi r^4}{8l\eta}\Delta P.
    \label{equ:puaz}
\end{equation}
\subsection{Эксперементальная установка}
\begin{figure}[h]
    \centering
    \includegraphics*[width = 0.8\linewidth]{ust.png}
    \caption{Экспеременатльная установка}
    \label{pic:ust}
\end{figure}
Схема экспериментальной установки изображена на Рис. \ref*{pic:ust}. Поток воздуха
под давлением, немного превышающим атмосферное, поступает через
газовый счётчик в тонкие металлические трубки. Воздух нагнетается
компрессором, интенсивность его подачи регулируется краном К. Трубки снабжены
съёмными заглушками на концах и рядом миллиметровых отверстий, к
которым можно подключать микроманометр. В рабочем состоянии открыта
заглушка на одной (рабочей) трубке, микроманометр подключён к двум её
выводам, а все остальные отверстия плотно закрыты пробками.

Перед входом в газовый счётчик установлен водяной U-образный
манометр. Он служит для измерения давления газа на входе, а также предохраняет
счётчик от выхода из строя. При превышении максимального избыточного
давления на входе счётчика ($\sim$  30 см вод. ст.) вода выплёскивается из трубки
в защитный баллон Б, создавая шум и привлекая к себе внимание экспериментатора.

