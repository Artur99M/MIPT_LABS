\section{Выводы}

Измерена зависимость коэффициента поверхностного натяжения воды в зависимости от температуры. Было замечено, что при увелечении температуры этот коффициент падает. Также исследованы зависимости теплоты образования единицы площади жидкости от температуры и полной поверхностной энергии пузыря воздуха от температуры. Зависимость коэффициента поверхностного натяжения воды от температуры получилась такая:
\[\sigma(T) = -0,1626T + 109,25 \text{ }\frac{\text{мH}}{\text{м}}.\]

Заметим, что при небольшом изменении температуры, значение коэффициента поверхностного натяжения не сильно изменяется, поэтому при нормальных условиях его можно принять равным $60 \text{ }\frac{\text{мH}}{\text{м}}$. Также есть температура, при которой энергия, необходимая для образования единицы площади, равна нулю. Но это происходит, когда при атмосферном давлении вся вода превратилась в пар.
