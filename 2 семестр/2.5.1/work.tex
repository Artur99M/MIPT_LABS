\section{Ход работы}
\subsection{Измерения}
1. Измерено максимальное давление спирта $\Delta P_\text{спирт}$ при  пробулькивании пузырьков воздуха через спирт. Все результаты оказались абсолютно одинаковыми, поэтому в данном экмперименте нет необходимости подсчитывать случайную погрешность. Показания манометра оказались равны $d = 41$ мм.

2. Измерен диаметр иглы, который оказался равен $D = (1,1 \pm 0,05)$ мм.

3. Игла погружена в воду и соприкасается с поверхностью. Высота торчащей сверху части оказалась равна $h_1 = 11,5 \pm 1$ мм. Высота столба, до которой поднялся манометр при повторении эксперимента, который был со спиртом, оказалась $d_1 = 112$ мм.

4. Игла опущена на дно и проведены все те же измерению, что и с иглой наверху: $h_2 = 5,5 \pm 1$ мм,  $d_2 = 188$ мм.

5. Далее повышалась температура. Результаты приведены в таблице:

\begin{center}
\begin{tabular}[h]{|c|c|}
\hline
    t, $^\circ C$ & d, мм\\
    \hline
    22 & 188\\
    \hline
   30 & 187 \\
 \hline
   35 & 185 \\
 \hline
   40 & 184 \\
 \hline
   45 & 182 \\
 \hline
   50 & 181 \\
 \hline
   55 & 179 \\
 \hline
   60 & 178 \\
 \hline
   65 & 175 \\
 \hline
\end{tabular}
\end{center}

\subsection*{Обработка}

6. Посчитаем $\Delta P_\text{спирт}$ для измерения проведенных в пункте 1:
\begin{equation}
  \Delta P_\text{спирт} = 80,9 \text{ Па}.
  \label{equ:spirt}
\end{equation}
Так как все измерения были одинаковыми, рассмотрим только систематическую погрешность, выванную тем, что манометр работает не идеально. Эту погрешность примем за единицу деления.

7. Посчитаем, исползуя (\ref*{equ:spirt}), диаметр иглы по формуле (\ref*{equ:laplas}):
\begin{equation}
  D_\text{экс} = \frac{4\sigma}{\Delta P_\text{спирт}} = (1,08 \pm 0,1)\text{ мм}.
\end{equation}
Заметим, что проведенные с помощью микроскопа измерения полностью лежат в области погрешности измерений с помощью спирта.

8. Посчитаем добавочное давление $\Delta P$ при погружении иглы:
\begin{equation}
  \Delta P = \rho g\Delta h = (59 \pm 7) \text{ Па}.
\end{equation}

9. Далее посчитаем соотвествующие всем случаям $\sigma(T)$:
\begin{center}
\begin{tabular}[h]{|c|c|}
  \hline
  $\sigma$, мПа $\cdot$ м & $T$, К\\
  \hline
   60.78 & 295.15 \\
 \hline
   60.24 & 303.15 \\
 \hline
   59.15 & 308.15 \\
 \hline
   58.61 & 313.15 \\
 \hline
   57.52 & 318.15 \\
 \hline
   56.98 & 323.15 \\
 \hline
   55.89 & 328.15 \\
 \hline
   55.35 & 333.15 \\
 \hline
   53.72 & 338.15 \\
 \hline

\end{tabular}
\end{center}

Получилось уравнение $\sigma(T) = -0.1626T + 109.25$.

Построим график:

\begin{figure}[h]
  \centering
  \includegraphics*[width = 0.5\linewidth]{p2.png}
\end{figure}

Случайная погрешность $\Delta\left(\frac{d\sigma}{dT}\right)_\text{случ} = 0,012$. Учитывая погрещность измерений манометра, равной одному делению, посчитаем систематическую погрешность. Усредним систематическую погрешность для простого расчета $\Delta\left(\frac{d\sigma}{dT}\right)_\text{сист} = 0,0025$.

Итого получилось: $\frac{d\sigma}{dT} = -0,1626 \pm 0,01 \text{ } \frac{\text{Н}}{\text{м}\cdot\text{К}}.$

10. Построим график $q = q(T) = -T\frac{d\sigma}{dT}$:

\begin{figure}[h]
  \centering
  \includegraphics*[width = 0.5\linewidth]{p3.png}
\end{figure}

Заметим, что получилась в точности прямая зависимость, что и ожидалось.

11. Построим график зависимости энергии U единицы площади, $U = \sigma - T\frac{d\sigma}{dT}$:

\begin{figure}[h]
  \centering
  \includegraphics*[width = 0.5\linewidth]{p4.png}
\end{figure}
В данном случае можно заметить, что была совершена некая ошибка, так как энергия получилась отрицательной, чего быть не может. Будем считать эту точку ошибочной.
