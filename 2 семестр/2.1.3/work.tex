\section{Ход работы}
\subsection{Измерения}

1. Некоторые данные:
\[L_1 = (570 \pm 5) \text{ мм -- начальная длина трубы,}\]
\[T = 296 \text{ К -- температура в комнате,}\]
\[L_2 = (800 \pm 1) \text{ мм -- длина трубы для 2 измерения.}\]

2. Для начала измерена зависимость длины трубы от возникновения резонанса
\begin{table}[h]
    \centering
    \begin{tabular}{|c|c|c|c|c|c|}
    \hline
    Частота, Гц & Длина, мм & Длина, мм & Длина, мм & Длина, мм & Длина, мм \\
    \hline
    2280 & 45 & 124 & 198 &&\\
    \hline
    2600 & 40 & 109 & 171 &&\\
    \hline
    2900 & 32 & 91 & 149 & 210 &\\
    \hline
    3200 & 39 & 94 & 146 & 202 &\\
    \hline
    3500 & 30 & 81 & 130 & 180 & 230 \\
    \hline
    2000 & 43 & 128 & 217 &&\\
    \hline
    \end{tabular}
    \caption{Измерения для воздуха}
\end{table}

\begin{table}[h]
    \centering
    \begin{tabular}{|c|c|c|c|c|c|}
    \hline
    Частота, Гц & Длина, мм & Длина, мм & Длина, мм & Длина, мм & Длина, мм \\
    \hline
    2000 & 180 & 113 & 45 &&\\
    \hline
    2300 & 196 & 138 & 80 & 20 &\\
    \hline
    2600 & 212 & 164 & 110 & 600 & 6 \\
    \hline
    2900 & 185 & 139 & 94 & 45 &\\
    \hline
    3200 & 200 & 154 & 112 & 70 & 30 \\
    \hline
    \end{tabular}
    \caption{Измерения для углекислого газа}
\end{table}

3. Также измерена зависимость резонансной частоты от температуры при постоянной длине трубы.

\begin{table}[h]
    \centering
    \begin{tabular}{|c|c|c|c|c|c|c|}
    \hline
    Температура, $^\circ C$ & Частота, Гц & Частота, Гц & Частота, Гц & Частота, Гц & Частота, Гц & Частота, Гц \\
    \hline
    27 & 203.8 & 452.5 & 664.3 & 879 & 1095.5 & 1314 \\
    \hline
    35 & 205 & 459 & 671 & 890 & 1110 & 1330 \\
    \hline
    45 & 209 & 464 & 683 & 903 & 1127 & 1351 \\
    \hline
    55 & 211 & 471 & 693 & 917 & 1144 & 1372 \\
    \hline
    \end{tabular}
    \caption{Измерения зависимости от температуры.}
\end{table}

\subsection{Обработка}

4. Для воздуха построим график L. Уравнения:
\[\Delta L(k) = 76.5k \text{ для }\nu = 2280 \text {Гц},\]
\[\Delta L(k) = 65.5k \text{ для }\nu = 2600 \text {Гц},\]
\[\Delta L(k) = 59.2k \text{ для }\nu = 2900 \text {Гц},\]
\[\Delta L(k) = 54.1k \text{ для }\nu = 3200 \text {Гц},\]
\[\Delta L(k) = 49.9k \text{ для }\nu = 3500 \text {Гц},\]
\[\Delta L(k) = 87k \text{ для }\nu = 2000 \text {Гц}.\]
\begin{figure}
    \centering
    \includegraphics*[width = \linewidth]{p3.png}
    \caption{Зависмость $\Delta L(k)$ для воздуха}
\end{figure}

5. Для углекислого газа построим графики. Уравнения:
\[\Delta L(k) = 67.5k \text{ для }\nu = 2000 \text {Гц},\]
\[\Delta L(k) = 58.6k \text{ для }\nu = 2300 \text {Гц},\]
\[\Delta L(k) = 51.6k \text{ для }\nu = 2600 \text {Гц},\]
\[\Delta L(k) = 46.5k \text{ для }\nu = 2900 \text {Гц},\]
\[\Delta L(k) = 42.4k \text{ для }\nu = 3200 \text {Гц}.\]
\begin{figure}
    \centering
    \includegraphics*[width = \linewidth]{p4.png}
    \caption{Зависмость $\Delta L(k)$ для углекислого газа}
\end{figure}

6. Значение скорости звука в воздухе получилось $(346 \pm 1)$ м/с. А для углекислого газа $(270 \pm 1)$ м/с.

7. Анализ измерений на второй установке тоже сводится к построению графика. Уравнения:
\[\Delta \nu(k) = 220k \text{ для }T = 27 ^\circ C,\]
\[\Delta \nu(k) = 222.7k \text{ для }T = 35 ^\circ C,\]
\[\Delta \nu(k) = 226k \text{ для }T = 45 ^\circ C,\]
\[\Delta \nu(k) = 230k \text{ для }T = 55 ^\circ C.\]
Заметим, что при повышении температуры график вращается против часовой стрелки.

\begin{figure}
    \centering
    \includegraphics*[width = \linewidth]{p5.png}
    \caption{Зависмость $\Delta \nu(k)$ для воздуха}
\end{figure}

8. По графику из пункта 7 определены $\frac{c}{2L_2} = (224.7 \pm 3) \text{Гц}$.
Таким образом $c = 360 \pm 4$ м/с.

9. Для воздуха $\gamma = (1.45 \pm 0.05)$, для углекислого газа $\gamma = 1.29 \pm 0.02$.
