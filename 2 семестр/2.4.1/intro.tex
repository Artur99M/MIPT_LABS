\section{Введение}
\subsection{Теоретическое введение}
\textbf{Цель работы:}
измерение давления насыщенного пара жидкости
при разной температуре; 2) вычисление по полученным данным теплоты испарения с помощью уравнения Клапейрона-Клаузиуса.

\textbf{В работе используются:}
термостат; герметический
сосуд,
заполненный исследуемой жидкостью; отсчётный микроскоп.


Испарением называется переход вещества из жидкого в газообразное состояние. Оно происходит на свободной поверхности жидкости. При испарении с поверхности вылетают молекулы, образуя над ней пар. Для выхода из жидкости молекулы должны преодолеть силы молекулярного сцепления. Кроме того, при испарении совершается работа против внешнего давления $P$, поскольку объём жидкости меньше объёма пара.
Не все молекулы жидкости способны совершить эту работу, а только те
из них, которые обладают достаточной кинетической энергией. Поэтому переход части молекул в пар приводит к обеднению жидкости быстрыми
молекулами, т. е. к её охлаждению. Чтобы испарение проходило без изменения температуры, к жидкости нужно подводить тепло. Количество теплоты, необходимое для изотермического испарения одного моля жидкости при внешнем давлении, равном упругости её насыщенных паров, называется молярной теплотой испарения (парообразования).

Теплоту парообразования жидкостей можно измерить непосредственно при помощи калориметра. Такой метод, однако, не позволяет получить точных результатов из-за неконтролируемых потерь тепла, которые
трудно сделать малыми. В настоящей работе для определения испарения применен косвенный метод, основанный на формуле теплоты
Клапейрона-Клаузиуса:
\begin{equation}
    \frac{dP}{dT} = \frac{L}{T\left(V_2 - V_1\right)}.
    \label{equ:L}
\end{equation}
Здесь Р -- давление насыщенного пара жидкости при температуре Т, T --  абсолютная температура жидкости
и пара, L — теплота испарения, $V_2$ -- объем пара, $V_1$ -- объем жидкости.

Обратимся теперь к V2, которое в дальнейшем будем обозначать
просто V. Объём V связан с давлением и температурой уравнением Ван-дер-Ваальса:
\begin{equation}
    \left(P + \frac{a}{V^2}\right)\left(V-b\right) = RT.
    \label{equ:VDV}
\end{equation}

В уравнении Ван-дер-Ваальса величиной $b$ следует пренебречь. Пренебрежение членом $a/V^2$ по сравнению с $P$ вносит ошибку менее 3\%. При давлении ниже атмосферного ошибки становятся ещё меньше. Таким образом, при давлениях ниже атмосферного уравнение Ван-дер-Ваальса для насыщенного пара мало отличается от уравнения Клапейрона. Положим поэтому

\begin{equation}
    V = \frac{RT}{P}.
    \label{equ:PVRT}
\end{equation}

Подставляя (\ref*{equ:PVRT}) в (\ref*{equ:L}) получаем:
\begin{equation}
    L = \frac{RT^2}{P}\frac{dP}{dT} = - R \frac{d(\ln P)}{d(1/T)}.
\end{equation}

\subsection{Эксперементальная установка}
\begin{wrapfigure}{r}{0.4\linewidth}
    \centering
    \includegraphics*[width = 0.7\linewidth]{p3.jpg}
    \caption{Схема установки для определения теплоты испарения}
    \label{pic:ust}
\end{wrapfigure}
Схема установки изображена на рис. \ref*{pic:ust}. Наполненный водой резервуар 1 играет роль термостата. Нагревание термостата
производится спиралью 2, подогреваемой электрическим током. Для охлаждения воды в термостате через змеевик 3 пропускается водопроводная вода. Вода в термостате перемешивается с воздухом, поступающим через трубку 4
Температура воды измеряется термометром 5. В термостат погружен запаянный прибор 6 с исследуемой жидкостью. Над ней находится насыщенный пар (перед заполнением прибора воздух из него был откачан). Давление насыщенного пара определяется по ртутному манометру, соединённому с исследуемым объёмом. Отсчёт показаний
манометра производится при помощи микроскопа.
