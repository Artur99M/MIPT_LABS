\section{Выводы}

Значение для удельной теплоты парообразования не совпадает с табличным даже в пределах погрешности. Это оправдывается наличием в воде примесей и растворенного воздуха. В случаях нагревания и охлаждения теплота испарения воды получилась разной, так как установка была не идеальной и температура воды могла получится не той, какой предполагала модель.

Получились результаты, которые очень близки к правде:
\[L_\text{нагр} = (51128 \pm 600) \text{ }\text{Дж},\]
\[L_\text{охл} = (48871 \pm 800) \text{ }\text{Дж}.\]
