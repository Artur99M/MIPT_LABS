\subsubsection{Измерения}
13. Для начала измерены или изучены в паспорте некоторые величины:
\begin{itemize}
    \item длина груза $x = 40$ мм,
    \item длина проволоки $L = (1750 \pm 2)$ мм,
    \item толщина проволоки $\delta = 2,0 \pm 0,1$ мм,
    \item масса превого груза $m_1 = 373$ г,
    \item масса второго груза $m_2 = 378$ г.
\end{itemize}

14. Измерим период колбаний:

\begin{tabular*}{0.915\textwidth}{|c|c|c|c|}
    \hline
    Расстоние центра до груза, см & Количество колбаний & Время, с & Период, с\\
    \hline
    6 & 20 & 63.40 & 3.17\\
    \hline
    5 & 20 & 57.54 & 2.88\\
    \hline
    4 & 20 & 52.19 & 2.61\\
    \hline
    3 & 20 & 48.21 & 2.41\\
    \hline
    2 & 20 & 42.87 & 2.14\\
    \hline
    1 & 20 & 39.26 & 1.96\\
    \hline

\end{tabular*}
\subsubsection{Обработка}
15. Посчитаем момент инерции относительно центра:
\begin{equation}
    I = (m_1 + m_2)(r + \frac x2).
\end{equation}

16. Выразим $f$ из (11):
\begin{equation}
f = 4\pi^2\frac I {T^2}.
\end{equation}

17. Объединяя (18) и (7) получаем:
\begin{equation}
    G = \frac{8\pi l\left(m_1 + m_2\right)\left(r + \frac x2\right)}{T^2\left(\frac{\delta}{2}\right)^4}.
\end{equation}

18. Подставляя разные значения в (19) получаем

\[G_6 = 262826442813.4,\]
\[G_5 = 279200330406.9,\]
\[G_4 = 290893690923.7,\]
\[G_3 = 284088284762.8,\]
\[G_2 = 287415778334.9,\]
\[G_1 = 257026705551.3.\]

19. Итог $G = 2.76 \cdot 10^{11} \pm 3.1 \cdot 10^{10}$.
