\section{Ход работы}
1. Для начала измерены веса всех пуль (в граммах):

\[m_1 = 0,507\]
\[m_2 = 0,506\]
\[m_3 = 0,516\]
\[m_4 = 0,512\]
\[m_5 = 0,512\]
\[m_6 = 0,510\]
\[m_7 = 0,508\]
\[m_8 = 0,503\]
\[m_9 = 0,508\]
\[m_{10} = 0,500.\]

Погрешность измерения $\Delta m =0,001$ г.

\subsection{Баллистический маятник, совершающий поступательное движение}
\subsubsection{Измерения}

2. Основные измерения для маятника:

масса маятника $M_1 = \left(2905\pm5\right)$ г,

длина нити маятника $L = (221\pm1)$ см.

3. Измерения амплитуды колебаний в мм (индекс массы элемента
соответствует индексу амплитуды):

\[l_1 = 11,5\]
\[l_2 = 11,5\]
\[l_3 = 12,0\]
\[l_4 = 11,25\]
\[l_5 = 11,25.\]

Погрешность измерений $\Delta l = 0,25$ мм.

\subsubsection{Обработка}

4. Закон сохранения импульса для пули и маятника, считая, что
масса маятника много больше массы пули:

\[mV = Mu,\]
\[V=\frac{Mu}{m}.\]

5. Закон сохранения энергии для маятника, где h -- максимальная
высота, на которую поднимется груз:

 \[\frac{Mu^2}{2} = Mgh,\]
\[u^2 = 2gh.\]

6. Высота подъема маятника выражается через угол $\alpha$ наибольшего
отклонения от вертикали:
\begin{center}
    $h = L \left(1 - \cos{\alpha}\right) = 2L \left(\sin^2{\frac{\alpha}{2}}\right)$,
    где $\sin{\alpha} = \frac{l}{L}$.
\end{center}

7. Итого из 4, 5, 6 получаем:
\[V = \frac Mm\sqrt\frac gL l.\]

8. Посчитаем скорости для каждой массы:

\begin{center}

$V_1 = 138,827144$ м/с,

$V_2 = 139,101506$ м/с,

$V_3 = 142,336425$ м/с,

$V_4 = 134,482902$ м/с,

$V_5 = 134,482902$ м/с.

\end{center}

9. Посчитаем погрешность:

\begin{center}

$\Delta V = \sqrt{\left(\frac{\partial V}{\partial M}\right)^2\cdot \left(\Delta M\right)^2 + \left(\frac{\partial V}{\partial m}\right)^2\cdot \left(\Delta m\right)^2 + \left(\frac{\partial V}{\partial L}\right)^2\cdot \left(\Delta L\right)^2 + \left(\frac{\partial V}{\partial l}\right)^2\cdot \left(\Delta l\right)^2}$

\end{center}

Для каждого измерения получилась погрешность $\Delta V \approx 3,4$ м/с.

10. Итоговый результат:

\begin{center}

$V_1 = (138,8 \pm 3,4)$ м/с,

$V_2 = (139,1 \pm 3,4)$ м/с,

$V_3 = (142,3 \pm 3,4)$ м/с,

$V_4 = (134,5 \pm 3,4)$ м/с,

$V_5 = (134,5 \pm 3,4)$ м/с,

\end{center}

11. Теперь сделаем вид, что все пули одинаковой массы. Вычислим скорость как среднее арифметическое:

\begin{center}

$V_{mid} = 137, 8$ м/с,

\end{center}
случайную погрешность как среднее квадратичное:
\begin{center}

$\delta V = \sqrt{\frac{1}{4\cdot5}\sum\left(V_{mid} - V_i\right)^2} = 1,5$ м/с,

\end{center}
полную погрешность:
\begin{center}

$\Delta V_{mid} = \sqrt{\delta V^2 + \Delta V^2} = 3,7$ м/с,

\end{center}
итог:
\begin{center}

$V = (137,8 \pm 3,7)$ м/с.

\end{center}

\subsection{Крутильный маятник}

\subsubsection{Измерения}

12. Измерим все, что нужно в эксперименте:

расстояние от подвеса до линейки $d = (126 \pm 1)$ см,

расстояние от оси до места попадания пули $r = (21 \pm 0,1)$ см,

масса первого груза $M_1 = (713,5 \pm 0,1)$ г,

масса второго груза $M_2 = (730,6 \pm 0,1)$ г,

расстояние от точки подвеса до грузов $R = (33,5 \pm 0,1)$ см,

период малых колебаний с грузами $T_1 =\frac{60}{19} = 6,3$ с,

период малых колбебаний без грузов $T_2 = \frac{60}{25} = 4,8$ с.

13. Измерим амплитуду колебаний для разных грузов (индекс амплитуды
соответсвтует индексу массы):
\begin{center}

$x_6 = 41$ мм,

$x_7 = 38$ мм,

$x_8 = 35$ мм,

$x_9 = 41$ мм,

$x_{10} = 41$ мм.

\end{center}

Погрешность измерений $\sigma = 1$ мм.

\subsubsection{Обработка}
14. При малых углах верно $\varphi \approx \frac{x}{2d}$,
поэтому получаем формулу
\[u = \frac{x\sqrt{kI}}{2dmr} = \frac{2\pi\left(M_1+M_2\right)R^2T_1}{T_{1}^2 - T^{2}_2}\cdot\frac{x}{2dmr}.\]

15. Посчитаем $M_1+M_2$:
\begin{center}
$\Delta M = \sqrt{\Delta M_1^2 +\Delta M_2^2} = 0,14$ г,

$M = (1444,1 \pm 0,1)$ г.
\end{center}

16. Формула погрешности получается:
\begin{center}

$\Delta u_i = \sqrt{\left(\frac{\partial V}{\partial M}\right)^2\cdot\Delta M^2+\left(\frac{\partial V}{\partial R}\right)^2\cdot\Delta R^2+\dots}$

\end{center}

17. Итоговые измерения скорости (индекс скорости соответствует индексу массы):

\begin{center}
$u_{6} = (174.7 \pm 4.6)$ м/с,

$u_{7} = (162.6 \pm 4.6)$ м/с,

$u_{8} = (151.2 \pm 4.6)$ м/с,

$u_{9} = (175.4 \pm 4.6)$ м/с,

$u_{10} = (178.2 \pm 4.7)$ м/с.
\end{center}

18. Пренебрегая тем, что пули имеют разную массу, а систематическая
погрешность практически одинаковая $\approx 4,6$ м/с, посчитаем
скорость пули, используя метод наименьших квадратов:

\begin{center}

$u_{mid} = \frac{\sum u_i} n = 168,4$ м/с,

$\delta u = \sqrt{\frac{1}{4\cdot5}\sum_{i = 6}^{10}(u_{mid} - u_i)^2} = 5,1$ м/с,

$\Delta u = \sqrt{\delta u^2 + \delta u_{sist}^2} = 6,9$ м/с,

$u = \left(168,4 \pm 6,9\right)$ м/с.
\end{center}
